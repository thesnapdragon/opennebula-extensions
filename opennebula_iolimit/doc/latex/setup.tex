Requirement of the script is the cgroup program, which is usually in the main Linux repository, so it can be installed by the command: Debian / Ubuntu: 
\begin{DoxyCode}
 # apt-get install cgroup-bin
\end{DoxyCode}
 Red Hat Enterprise / Fedora Linux / Suse Linux / Cent OS: 
\begin{DoxyCode}
 # yum install libcgroup.i686
\end{DoxyCode}
 or 
\begin{DoxyCode}
 # yum install libcgroup.x86_64
\end{DoxyCode}


Cgroup can set IO limits only on block devices. Recommended to use LVM, for instance: \par
 shared lvm patch: \href{http://dev.opennebula.org/issues/1341}{\tt http://dev.opennebula.org/issues/1341}

{\bfseries Hook configuration} \par
 COMMON: Path to scripts\_\-common.sh, used for log functions \par
 DEVICEDIR: Path to the mounted VM devices \par
 CGROUP: Path to cgroup directory

For right behavior 'IOLIMIT' custom variable must be present in VM's template. 'IOLIMIT' variable specifies upper limit on write-\/ and read rate to the device. IO rate is specified in bytes per second.\par
 Options for adding custom variables: \begin{DoxyItemize}
\item In Sunstone by VM Template Wizard in section 'Add custom variables' \item Manually by editing a template\end{DoxyItemize}
Example for template with 1024 byte/sec IO limit: \par
 
\begin{DoxyCode}
 CPU="1"
 DISK=[
     IMAGE="ttylinux - kvm",
     IMAGE_UNAME="oneadmin" ]
 FEATURES=[
     PAE="no" ]
 IOLIMIT="1024"
 MEMORY="512"
 NAME="example"
 OS=[
     ARCH="x86_64",
     BOOT="hd" ]
 RAW=[
     TYPE="kvm" ]
 TEMPLATE_ID="1"\n
\end{DoxyCode}


After setting 'IOLIMIT' variables in, a VM hook must be defined in \$ONE\_\-LOCATION/etc/oned.conf: 
\begin{DoxyCode}
 VM_HOOK = [
      name      = "iolimit",
      on        = "RUNNING",
      command   = "/var/tmp/one/hooks/iolimit.sh",
      arguments = "$TEMPLATE",
      remote    = "YES" ]
\end{DoxyCode}
 